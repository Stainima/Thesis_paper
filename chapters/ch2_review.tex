\chapter{GENERAL REVIEW OF LITERATURE}

\subsection{What are Grammars?}

Grammars both in the natural and artifical language can be defined as a set of 

Quesitons:
What is ARVADA?

Waht are parsers?

What are context free grammar?

Why do a replication study?

\subsection{Why C?}

In the original study \cite{kulkarniLearningHighlyRecursive2021}, the ARVADA algorithm was implemented in Python. When compared to GLADE \cite{bastaniSynthesizingProgramInput}, which was implemented in Java, ARVADA exhibited a slower average runtime across all benchmarks. As stated in the study, this could be attributed to the natural runtime disadvantage of Python compared to Java.

\vspace{\baselineskip}
In a comparative study, \enquote{A Pragmatic Comparison of Four Different Programming Languages} \cite{aliPragmaticComparisonFour2021}, it was found that if speed and efficiency were important in an implementation, C was a better options compared to Python. C, being a mid-level, statically typed, structured language that runs under a compiler, will always be faster than a dynamic language run under an interpreter such as Python \cite{kumarPythonLanguageComparison2022}. Along with being a structured language, C also comes with only essential features. These limited features contribute to its efficiency but also introduce a higher level of complexity compared to Python \cite{aliPragmaticComparisonFour2021}\cite{kumarPythonLanguageComparison2022}.

\vspace{\baselineskip}
Hence, with the aim of replicating and improving upon the runtime bottleneck presented by Python—while acknowledging the rise in complexity C introduces compared to Python—C was chosen as the language of implementation.

Why ARVADA?

Why is learning highly input grammar important?

What is the problem statement?

What is GLADE?

What are other Similar works done?

What is the work done in this field after ARVADA?



