\usepackage[
    top=1in,
    bottom=1in,
    inner=0.8in,
    outer=1.2in,
    % showframe
]{geometry}
\raggedbottom
\pdfpagewidth=\paperwidth 
\pdfpageheight=\paperheight

\usepackage[utf8]{inputenc}

\emergencystretch=1em

%%% Bibliography %%%
\usepackage[backend=biber,style=ieee]{biblatex}
\addbibresource{thesis_references.bib}
\AtEveryBibitem{
    %\clearfield{doi}
    \clearfield{url}
    \clearfield{urlyear}
    \clearfield{urlmonth}
    \clearfield{note}
    % \clearfield{isbn}
    \clearfield{issn}
    \clearfield{eprint}
    \clearfield{eprinttype}
    \clearfield{langid}
}

%%% Image packages %%%
\usepackage{graphicx}
\graphicspath{ {./img/} }
\usepackage{subcaption} % For subfigures

% For drawing neural networks
\usepackage{neuralnetwork}
\newcommand{\xin}[2]{$x_#2$}
\newcommand{\xout}[2]{$\hat x_#2$}

%%% Tables %%%
\usepackage{booktabs}
\usepackage{multirow}
\usepackage{array}
\usepackage{rotating}

%%% Text %%%
\usepackage{xurl} % line breaks for urls
\usepackage{csquotes} % For quoting

\usepackage{hyperref}
\hypersetup{
    colorlinks=true,
    citecolor=black,%blue!80,
    linkcolor=black, 
    filecolor=black,    
    urlcolor=black,
    linktoc=all, 
    unicode=true
}

\usepackage{setspace} % line spacing
\usepackage{lipsum} % lorem ipsum paragraphs

\usepackage[svgnames]{xcolor} % colors

\usepackage[font={small},labelfont={bf},margin=12ex,skip=15pt]{caption} % Makes bold labelled and smaller font captions.
\captionsetup[table]{belowskip=5pt}  % Set above and below padding

\renewcommand{\thefootnote}{\alph{footnote}} % Lettered footnotes 

\usepackage[acronym, nonumberlist, nopostdot]{glossaries} %%% List of Abbreviations
\makeglossaries

\usepackage{titlesec}
\titleformat*{\paragraph}{\itshape\mdseries} % Change \paragraph styling

% Poetry
\usepackage{verse}
\newcommand{\attrib}[1]{%
\nopagebreak{\raggedleft\footnotesize #1\par}
}

%%% Math %%%
\usepackage{amsmath}
\usepackage{bm} % Bold vectors
\usepackage{dsfont} % Literally only here for the indicator function

%%% TikZ %%%
\usepackage{tikz}
\usetikzlibrary{shapes, arrows.meta, positioning, calc}