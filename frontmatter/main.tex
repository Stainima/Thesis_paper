%%%%%%%% TITLE PAGE %%%%%%%%%%%%
\newgeometry{
    inner=2.5cm,
    outer=2.5cm,
    top=2.5cm,
    bottom=2.5cm
}
\onehalfspacing
\begin{titlepage}
    \vspace*{1cm}

    \begin{flushleft}
        \begin{center}
            \includegraphics[width=0.35\textwidth]{img/pre_chapters/usyd_logo_full.png}
    
        \vspace{3cm} 
    
        {\LARGE \textbf{\enquote{Learning Highly Recursive Input Grammars}:\\[3pt]A Replication Study}}
    
        \vspace{1.5cm} 
        % \vfill
        
        {\Large Sulav Malla} 

        % \vspace{1.5cm}

        % \textit{A thesis presented as part of the requirements for the conferral of the degree of\\}
        % \textit{A dissertation submitted in partial fulfillment of the requirements for the degree of\\[0.2em]}
        % {\large BACHELOR OF ENGINEERING HONOURS (SOFTWARE)}
        % Bachelor of Engineering Honours (Software)
        
        % This thesis is presented as part of the requirements for the conferral of the degree:\\
        % \vspace{1em}
        % \textsc{Bachelor of Engineering Honours (Software)} 

        \vfill
        
        \textit{A thesis presented in partial fulfilment of the requirements for the degree of\\[3pt]}
        Bachelor of Engineering Honours (Software)
        
        \vspace{2.75cm}
    
        \textbf{Supervisors:}\\
        Rahul Gopinath, The University of Sydney
    
        \vspace{0.8cm}

        School of Electrical and Computer Engineering\\
        The University of Sydney

        \vspace{0.8cm}
        
        November 04, 2025 
        
        \vspace{2cm}
        
        \end{center}
        
    \end{flushleft}

\end{titlepage}
\restoregeometry
\onehalfspacing

%%%%%%%%%%%%%%%% COPYRIGHT %%%%%%%%%%%%%%%%%%%%
\clearpage\null\thispagestyle{empty}
\begin{center}
    \vspace*{\fill}
    
    % \textcopyright \ 2025 Sulav Malla \\
    % All Rights Reserved\\ \vspace{1em}
    
    No part of this work may be reproduced, stored in a retrieval system, or transmitted \\
    in any form or by any means, electronic, mechanical, photocopying, or otherwise, \\
    without the prior permission of the author or The University of Sydney.
\end{center}

%%%%%%%%%%%%%%%% ABSTRACT %%%%%%%%%%%%%%%%%%%%
\chapter*{Abstract}
\addcontentsline{toc}{chapter}{Abstract}
Knowing the source code of a program greatly aids program comprehension, testing techniques such as fuzzing, optimisation, and debugging. However, due to various restrictions and other external factors, accessing source code is often not possible. To address this limitation, recent research has focused on inferring input grammars and execution behaviour directly from valid program inputs. The ARVADA algorithm, developed by Lemieux C. and Kulkarni N. at UCB in 2021, is a black-box approach designed to infer the grammar of a program using only valid inputs and a black-box oracle.

Motivated by the lack of formal guarantees and concerns about the selectiveness of the original study, this thesis attempts to re-implement the ARVADA algorithm in a clean-room environment using the C programming language. Although a complete implementation was not achieved, the work highlights the challenges faced and insights gained during the re-implementation. These include the inherent complexity of ARVADA, the difficulty of implementing such an algorithm in C, and the limited clarity in the explanation provided in the original paper.


%%%%%%%%%%%%%%%% Table of Contents %%%%%%%%%%%%%%%%%%%%
\newpage\thispagestyle{empty}
\tableofcontents
\listoffigures


